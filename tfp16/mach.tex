\section{$\mathcal{\mathcal{C} \mskip -4mu \mathcal{E}}$ Machine} \label{sec:mach}

Using the calculus of cactus environments defined in the previous section, we
derive an abstract machine: the $\mathcal{\mathcal{C} \mskip -4mu \mathcal{E}}$ machine. The syntax
and semantics are defined in Figure~\ref{fig:CEM}. 

\begin{figure*}
\textbf{Syntax}
\begin{align*}
\tag{State} s &::= \langle c, \sigma, \mu \rangle \\
\tag{Term} t &::= i \; | \; \lambda t \; | \; t \; t  \\
\tag{Variable} i &\in \mathbb{N}  \\
\tag{Closure} c &::= t [l] \\
\tag{Value} v &::= \lambda t[l] \\
\tag{Heap} \mu &::= \epsilon \; | \; \mu [ l \mapsto \rho ] \\
\tag{Environment} \rho &::= \bullet \; | \; c \cdot l \\
\tag{Context} \sigma &::= \square \; | \; \sigma \; c \;  | \; \sigma \; u \\
\tag{Location} l,u,f &\in \mathbb{N}
\end{align*}
\textbf{Semantics}
\begin{align*}
\tag{Upd}
\langle v,  \sigma \; u , \mu \rangle 
  &\rightarrow_{\mathcal{\mathcal{C} \mskip -4mu \mathcal{E}}}
\langle v, \sigma, \mu(u \mapsto v \cdot l) \rangle  
\; \textnormal{where} \; c \cdot l = \mu(u) \\
\tag{Lam}
\langle \lambda t[l], \sigma \; c, \mu \rangle 
  &\rightarrow_{\mathcal{\mathcal{C} \mskip -4mu \mathcal{E}}}
\langle t[f], \sigma, \mu[f \mapsto c \cdot l]\rangle f \not \in \textnormal{dom}(\mu)  \\
\tag{App}
\langle t \; t'[l], \sigma, \mu \rangle
  &\rightarrow_{\mathcal{\mathcal{C} \mskip -4mu \mathcal{E}}}
\langle t[l], \sigma \; t'[l], \mu \rangle \\
\tag{Var1}
\langle 0[l], \sigma, \mu \rangle
  &\rightarrow_{\mathcal{\mathcal{C} \mskip -4mu \mathcal{E}}}
\langle c, \sigma \; l , \mu \rangle 
\; \textnormal{where} \; c \cdot l' = \mu(l)\\
\tag{Var2}
\langle i[l], \sigma, \mu \rangle
  &\rightarrow_{\mathcal{\mathcal{C} \mskip -4mu \mathcal{E}}}
\langle (i-1)[l'], \sigma, \mu \rangle
\; \textnormal{where} \; c \cdot l' = \mu(l)
\end{align*}
\caption{Syntax and semantics of the $\mathcal{\mathcal{C} \mskip -4mu \mathcal{E}}$ machine.}
\label{fig:CEM}
\end{figure*}

The machine operates as a small-step implementation of the calculus, extended
only with a context to implement the updates from the NVar subderivation
($\sigma \; u$) and the operands from the NEval subderivation ($\sigma \; c$).
Much like the calculus, a term $t$ is inserted into an initial state $\langle
t[0], \sigma, \epsilon[0\mapsto\bullet]$. On the update rule, the current
closure is a value, and there is an update marker as the outermost context.
This implies that a variable was entered and that the current closure
represents the corresponding value for that variable. Thus, we update the
location $u$ that the variable entered, replacing whatever term was entered
with the current closure. The Lam rule takes an argument off the context and
binds it to a variable, allocating a fresh heap location for the bound
variable. This ensures that every instance of the variable will point to this
location, and thus the bound term will be evaluated at most once. The App rule
simply pushes an argument term in the current environment. The Var1 rule enters
the closure pointed to by the environment location, while the Var2 rule
traverses the cactus environment to locate the correct closure.  

To get some intuition for the $\mathcal{\mathcal{C} \mskip -4mu \mathcal{E}}$
machine and how it works, consider Figure~\ref{fig:state}, evaluation of the
term $(\lambda a.(\lambda b.b \; a) \lambda c.c \; a) \; ((\lambda i.i)
\lambda j.j)$, which is $(\lambda(\lambda0\;1)\;\lambda0\;1)\;((\lambda0)\;
\lambda0)$ with deBruijn indices.

\begin{figure}
\begin{align*}
&\langle (\lambda(\lambda0\;1)\;\lambda0\;1)\;((\lambda0)\;\lambda0)[0],\square,\epsilon[0\mapsto\bullet]\rangle\\
&\rightarrow_{\mathcal{\mathcal{C} \mskip -4mu \mathcal{E}}}\langle \lambda(\lambda0\;1)\;\lambda0\;1[0],\square (\lambda0)\;\lambda0[0],\epsilon[0\mapsto\bullet]\rangle\\ 
&\rightarrow_{\mathcal{\mathcal{C} \mskip -4mu \mathcal{E}}}\langle (\lambda0\;1)\;\lambda0\;1[1],\square,\epsilon[0\mapsto\bullet][1\mapsto(\lambda0)\;\lambda0[0]\cdot0]\rangle\\ 
&\rightarrow_{\mathcal{\mathcal{C} \mskip -4mu \mathcal{E}}}\langle \lambda0\;1[1],\square \lambda0\;1[1],\epsilon[0\mapsto\bullet][1\mapsto(\lambda0)\;\lambda0[0]\cdot0]\rangle\\ 
&\rightarrow_{\mathcal{\mathcal{C} \mskip -4mu \mathcal{E}}}\langle 0\;1[2],\square,\epsilon[0\mapsto\bullet][1\mapsto(\lambda0)\;\lambda0[0]\cdot0][2\mapsto\lambda0\;1[1]\cdot1]\rangle\\ 
&\rightarrow_{\mathcal{\mathcal{C} \mskip -4mu \mathcal{E}}}\langle 0[2],\square 1[2],\epsilon[0\mapsto\bullet][1\mapsto(\lambda0)\;\lambda0[0]\cdot0][2\mapsto\lambda0\;1[1]\cdot1]\rangle\\ 
&\rightarrow_{\mathcal{\mathcal{C} \mskip -4mu \mathcal{E}}}\langle \lambda0\;1[1],\square 1[2] 2,\epsilon[0\mapsto\bullet][1\mapsto(\lambda0)\;\lambda0[0]\cdot0][2\mapsto\lambda0\;1[1]\cdot1]\rangle\\ 
&\rightarrow_{\mathcal{\mathcal{C} \mskip -4mu \mathcal{E}}}\langle \lambda0\;1[1],\square 1[2],\epsilon[0\mapsto\bullet][1\mapsto(\lambda0)\;\lambda0[0]\cdot0][2\mapsto\lambda0\;1[1]\cdot1]\rangle\\ 
&\rightarrow_{\mathcal{\mathcal{C} \mskip -4mu \mathcal{E}}}\langle 0\;1[3],\square,\epsilon[0\mapsto\bullet][1\mapsto(\lambda0)\;\lambda0[0]\cdot0][2\mapsto\lambda0\;1[1]\cdot1][3\mapsto1[2]\cdot1]\rangle\\ 
&\rightarrow_{\mathcal{\mathcal{C} \mskip -4mu \mathcal{E}}}\langle 0[3],\square 1[3],\epsilon[0\mapsto\bullet][1\mapsto(\lambda0)\;\lambda0[0]\cdot0][2\mapsto\lambda0\;1[1]\cdot1][3\mapsto1[2]\cdot1]\rangle\\ 
&\rightarrow_{\mathcal{\mathcal{C} \mskip -4mu \mathcal{E}}}\langle 1[2],\square 1[3] 3,\epsilon[0\mapsto\bullet][1\mapsto(\lambda0)\;\lambda0[0]\cdot0][2\mapsto\lambda0\;1[1]\cdot1][3\mapsto1[2]\cdot1]\rangle\\ 
&\rightarrow_{\mathcal{\mathcal{C} \mskip -4mu \mathcal{E}}}\langle 0[1],\square 1[3] 3,\epsilon[0\mapsto\bullet][1\mapsto(\lambda0)\;\lambda0[0]\cdot0][2\mapsto\lambda0\;1[1]\cdot1][3\mapsto1[2]\cdot1]\rangle\\ 
&\rightarrow_{\mathcal{\mathcal{C} \mskip -4mu \mathcal{E}}}\langle (\lambda0)\;\lambda0[0],\square 1[3] 3 1,\epsilon[0\mapsto\bullet][1\mapsto(\lambda0)\;\lambda0[0]\cdot0][2\mapsto\lambda0\;1[1]\cdot1][3\mapsto1[2]\cdot1]\rangle\\ 
&\rightarrow_{\mathcal{\mathcal{C} \mskip -4mu \mathcal{E}}}\langle \lambda0[0],\square 1[3] 3 1 \lambda0[0],\epsilon[0\mapsto\bullet][1\mapsto(\lambda0)\;\lambda0[0]\cdot0][2\mapsto\lambda0\;1[1]\cdot1][3\mapsto1[2]\cdot1]\rangle\\ 
&\rightarrow_{\mathcal{\mathcal{C} \mskip -4mu \mathcal{E}}}\langle 0[4],\square 1[3] 3 1,\epsilon[0\mapsto\bullet][1\mapsto(\lambda0)\;\lambda0[0]\cdot0][2\mapsto\lambda0\;1[1]\cdot1][3\mapsto1[2]\cdot1][4\mapsto\lambda0[0]\cdot0]\rangle\\ 
&\rightarrow_{\mathcal{\mathcal{C} \mskip -4mu \mathcal{E}}}\langle \lambda0[0],\square 1[3] 3 1 4,\epsilon[0\mapsto\bullet][1\mapsto(\lambda0)\;\lambda0[0]\cdot0][2\mapsto\lambda0\;1[1]\cdot1][3\mapsto1[2]\cdot1][4\mapsto\lambda0[0]\cdot0]\rangle\\ 
&\rightarrow_{\mathcal{\mathcal{C} \mskip -4mu \mathcal{E}}}\langle \lambda0[0],\square 1[3] 3 1,\epsilon[0\mapsto\bullet][1\mapsto(\lambda0)\;\lambda0[0]\cdot0][2\mapsto\lambda0\;1[1]\cdot1][3\mapsto1[2]\cdot1][4\mapsto\lambda0[0]\cdot0]\rangle\\ 
&\rightarrow_{\mathcal{\mathcal{C} \mskip -4mu \mathcal{E}}}\langle \lambda0[0],\square 1[3] 3,\epsilon[0\mapsto\bullet][1\mapsto\lambda0[0]\cdot0][2\mapsto\lambda0\;1[1]\cdot1][3\mapsto1[2]\cdot1][4\mapsto\lambda0[0]\cdot0]\rangle\\ 
&\rightarrow_{\mathcal{\mathcal{C} \mskip -4mu \mathcal{E}}}\langle \lambda0[0],\square 1[3],\epsilon[0\mapsto\bullet][1\mapsto\lambda0[0]\cdot0][2\mapsto\lambda0\;1[1]\cdot1][3\mapsto\lambda0[0]\cdot1][4\mapsto\lambda0[0]\cdot0]\rangle\\ 
&\rightarrow_{\mathcal{\mathcal{C} \mskip -4mu \mathcal{E}}}\langle 0[5],\square,\epsilon[0\mapsto\bullet][1\mapsto\lambda0[0]\cdot0][2\mapsto\lambda0\;1[1]\cdot1][3\mapsto\lambda0[0]\cdot1][4\mapsto\lambda0[0]\cdot0][5\mapsto1[3]\cdot0]\rangle\\ 
&\rightarrow_{\mathcal{\mathcal{C} \mskip -4mu \mathcal{E}}}\langle 1[3],\square 5,\epsilon[0\mapsto\bullet][1\mapsto\lambda0[0]\cdot0][2\mapsto\lambda0\;1[1]\cdot1][3\mapsto\lambda0[0]\cdot1][4\mapsto\lambda0[0]\cdot0][5\mapsto1[3]\cdot0]\rangle\\ 
&\rightarrow_{\mathcal{\mathcal{C} \mskip -4mu \mathcal{E}}}\langle 0[1],\square 5,\epsilon[0\mapsto\bullet][1\mapsto\lambda0[0]\cdot0][2\mapsto\lambda0\;1[1]\cdot1][3\mapsto\lambda0[0]\cdot1][4\mapsto\lambda0[0]\cdot0][5\mapsto1[3]\cdot0]\rangle\\ 
&\rightarrow_{\mathcal{\mathcal{C} \mskip -4mu \mathcal{E}}}\langle \lambda0[0],\square 5 1,\epsilon[0\mapsto\bullet][1\mapsto\lambda0[0]\cdot0][2\mapsto\lambda0\;1[1]\cdot1][3\mapsto\lambda0[0]\cdot1][4\mapsto\lambda0[0]\cdot0][5\mapsto1[3]\cdot0]\rangle\\ 
&\rightarrow_{\mathcal{\mathcal{C} \mskip -4mu \mathcal{E}}}\langle \lambda0[0],\square 5,\epsilon[0\mapsto\bullet][1\mapsto\lambda0[0]\cdot0][2\mapsto\lambda0\;1[1]\cdot1][3\mapsto\lambda0[0]\cdot1][4\mapsto\lambda0[0]\cdot0][5\mapsto1[3]\cdot0]\rangle\\ 
&\rightarrow_{\mathcal{\mathcal{C} \mskip -4mu \mathcal{E}}}\langle \lambda0[0],\square,\epsilon[0\mapsto\bullet][1\mapsto\lambda0[0]\cdot0][2\mapsto\lambda0\;1[1]\cdot1][3\mapsto\lambda0[0]\cdot1][4\mapsto\lambda0[0]\cdot0][5\mapsto\lambda0[0]\cdot0]\rangle\\ 
\end{align*}
\caption{$\mathcal{\mathcal{C} \mskip -4mu \mathcal{E}}$ machine example.
Evaluation of $(\lambda(\lambda0\;1)\;\lambda0\;1)\;((\lambda0)\;\lambda0)$}
\label{fig:state}
\end{figure}

\subsection{Correctness}
We prove that the reflexive transitive closure of the $\mathcal{\mathcal{C} \mskip -4mu \mathcal{E}}$ machine
step relation evaluates to a value and heap and empty context iff
$\xrightarrow{}_{N}$ evaluates to the same value and heap.

{\thm \textnormal{(Equivalence)} $$(c, \mu) \rightarrow_{N} (v, \mu') \;
\leftrightarrow \; \langle c, \square, \mu \rangle \xrightarrow{*
}_{\mathcal{\mathcal{C} \mskip -4mu \mathcal{E}}} \langle v, \square, \mu' \rangle $$} 

By induction on the $\rightarrow_{N}$ step relation for one direction, and
induction on the reflexive transitive closure of the
$\rightarrow_{\mathcal{\mathcal{C} \mskip -4mu \mathcal{E}}}$ step relation for
the other.

