\section{Discussion} \label{sec:discussion}

This section reflects on the chapter, including threats to validity, future
work, related work, and general discussion of the results.

One thing that is important to communicate is the difficulty of writing
comprehensible proofs. The reader is discouraged from attempting to understand
the proofs in any way by reading the Coq tactic source code. While I attempted
to keep the definitions and lemmas as clean and comprehensible as possible, I
found it extremely difficult to do the same with tactics. Partially this may be
a failure on my part to become more familiar with the tactic language of Coq,
but I suspect that the imperative nature of tactic proofs prevents composability
of tactic meta-programs. 

Another lesson was the importance of good induction principles. For example, in
Section~\ref{sec:threats}, we discuss the issue of only proving the
implication of correctness in one direction. This is effectively a product of
the power of the inductive properties of high level semantics, which makes them
so much easier to reason about. Indeed, this lesson resonates with the purpose
of the chapter, which is that we'd like to reason about high level semantics,
because they are so much easier to reason about due to their pleasant inductive
properties, and have that reasoning preserved through compilation. 

\subsection{Threats to Validity} \label{sec:threats}

There are a few potential threats to validity that we address in this section. The
first is the one mentioned in Section~\ref{sec:background}, that we only show
that our compiler is correct in the case of termination of the source semantics.
In other words, if the source semantics doesn't terminate, the theorem says
nothing about how the compiled code behaves. This means that we could have a
compiled program that terminates when the source semantics does not terminate.

One argument in defense of verification presented here is that \emph{we
generally only care about preservation of semantics for preserving reasoning
about our programs}. In other words, if we have a program that we can't reason
about, and therefore may not terminate, we care less about having a proof that
semantics are preserved.  Of course, this is a claim about most uses of program
analysis. There are possible analyses that could say things along the lines of
\emph{if} the source program terminates, then we can conclude $x$. These cases
are rare, and therefore the provided proof of correctness can still be applied
to most use cases.  

Another potential threat to validity is the use of a high level instruction
machine language. While we claim that its high level and simplicity should make
it possible to show that a set of real ISAs implement this instruction machine,
we haven't formally verified this step. This would make for valuable future
work, and nothing in the design of our high level instruction machine prevents
such work.

As a dual to the issue of a high level instruction machine language, some
readers may take issue with calling lambda calculus with de Bruijn indices an
"input language". Indeed, we do not advocate writing programs in such a
language. Still, the conversion between lambda calculus with named variables and
lambda calculus with de Bruijn indices is a well understood topic, and 
it would distract from the presentation of the verified compiler
provided here. Indeed, as noted below, semantics using named variables and
substitution can be hard to get right \cite{breitnerthesis, nakata2009small}, so
we stand by our decision to use a semantics based on lambda calculus with de
Bruijn indices. One potential approach for future work would be to prove that a
call-by-name semantics using substitution is equivalent to Curien's calculus of
closures, which when combined with a proof that our call-by-need implements our
call-by-name, would prove the compiler implements the semantics of a standard
lambda calculus with named variables.

A third threat to the validity of this work is the question of whether 
we have really proved that we have implemented \emph{call-by-need}. The question
naturally arises of what exactly it means to prove an implementation of
call-by-need is correct. There are certainly well-established semantics
\cite{launchburynatural, ariola1995call}, so one option would be to directly
prove that the $\mathcal{CE}$ semantics implements one of those existing
semantics.  Unfortunately, recent work has shown that both of these have small
issues that arise when formalized that require fixes. Indeed, we did stray down
this path and rediscovered one of these issues which has been previously described
in the literature \cite{nakata2009small}. This raises the question of whether or
not semantics that aren't obviously correct are a good base for what it means to
be a call-by-need semantics. Instead, we have chosen to
relate our call-by-name semantics formally to a semantics that is obviously
correct, Curien's calculus of closures. Along with the small modification
required for memoization of results, we hope that we have convinced the reader
that it is \emph{extremely likely} that the memoization of results is correct.
Of course, further evidence such as examples of correct evaluation would go
further to convince the reader, and for that we encourage readers to
play with a toy implementation at \url{https://github.com/stelleg/cem_pearl}.
Finally, a more convincing result would be a proof that the call-by-need
semantics implement the call-by-name semantics. 

Yet another threat to validity is our approach (or lack of approach) to
heap-reuse. For simplicity, we have assumed that our fresh locations are fresh
with respect to all existing bindings in the heap. Of course, this is
unsatisfactory when compared to real implementations. It would be preferable to
have our freshness constraint relaxed to only be fresh with respect to live
bindings on the heap. This modification should be possible, at the cost of
increased complexity in the proofs. 

\subsection{Future Work}

In addition to some of the future work discussed as ways of addressing issues 
in Section~\ref{sec:threats}, there are some additional features that we think
make for exciting areas of future work. 

One such area is reasoning about preservation of operational properties such as
time and space requirements. This would enable reasoning about time and space
properties at the source level and ensuring that these are preserved through
compilation. In addition, there is the possibility of verified optimizations,
where one can prove that some optimizations are both \emph{correct}, in that
they provably preserve semantics, and \emph{true} optimizations, in that they
only improve performance with respect to some performance model. By defining a
baseline compiler and proving that it preserved operational properties such as
time and space usage, one would have a good platform for which to apply this
class of optimizations, resulting in a full compiler that verifiably preserves 
bounds on time and space consumption. As with correctness, reasoning about
operational properties is often likely to be easier in the context of the
easy-to-reason-about high level semantics, and having that reasoning provably
preserved would be extremely valuable. 

Another exciting area of future work is powerful proofs of type preservation
through compilation.  While there has been existing work on type-preserving
compilers, fully verified compilers like this one provide such a strong property
that type-safety should fall out directly.

One useful feature of Coq is the ability to extract Coq programs out to other
implementations, e.g., Haskell. This raises the possibility of extracting the
verified compiler out to a Haskell implementation that could be incorporated
into GHC, providing a path towards a verified Haskell compiler. 

\subsection{Relation to Ariola et al.}

As mentioned above, to improve the relation to existing work, we could relate
the semantics to the operational semantics of Ariola et al., seen in
Figure~\ref{fig:cbn} ~\cite{ariola1995call}. Because I spent a large amount of
time attempting this, I use this section to discuss challenges to this
approach. 

One approach I tried was to show bisimulation between $\rightarrow_{N}$ and
$\Downarrow$. Unfortunately, as mentioned in Section~\ref{sec:threats}, there is
an issue I discovered when formalizing this semantics. Essentially, the rules
for variable freshness are insufficient. To fix this, one has to modify the
rules to ensure global freshness. This modification is shown in
Figure~\ref{fig:cbn_fixed}. While it makes the semantics more complicated, a
significant challenge for proving bisimulation comes from the complexity of
managing the relation between de Bruijn terms and standard named terms through a
mutating heap. 

\begin{figure}
\begin{align*}
\tag{Heap} \Phi, \Psi, \Upsilon &::= x_1 \mapsto t_1, \dots, x_n \mapsto t_n
\end{align*}
\begin{align*}
\tag{Id} \inference
{\langle \Phi \rangle t \Downarrow \langle \Psi \rangle \lambda x.t'}
{\langle \Phi, x \mapsto t, \Upsilon \rangle x \Downarrow \langle \Psi, x
\mapsto \lambda x.t', \Upsilon \rangle \lambda x.t'}
\end{align*}
\begin{align*}
\tag{Abs} \inference 
{}
{\langle \Phi \rangle \lambda x . t \Downarrow \langle \Phi \rangle \lambda x.t}
\end{align*}
\begin{align*}
\tag{App} \inference
{\langle \Phi \rangle t_l \Downarrow \langle \Psi \rangle \lambda 
x.t_n \\ \langle \Psi, x' \mapsto t_m \rangle [x'/x]t_n \Downarrow \langle
\Upsilon \rangle \lambda y.t'}
{\langle \Phi \rangle t_l \; t_m \Downarrow \langle \Upsilon \rangle \lambda y.t'}
\end{align*}
\caption{Ariola et al.'s Operational Semantics}
\label{fig:cbn}
\end{figure}


\begin{figure}
\begin{align*}
\tag{Heap} \Phi, \Psi, \Upsilon &::= x_1 \mapsto t_1, \dots, x_n \mapsto t_n
\end{align*}
\begin{align*}
\tag{Id} \inference
{\langle \Phi, x \mapsto t, \Upsilon \rangle t \Downarrow \langle \Phi', x \mapsto t, \Upsilon' \rangle \lambda x.t'}
{\langle \Phi, x \mapsto t, \Upsilon \rangle x \Downarrow 
 \langle \Phi', x \mapsto \lambda x.t', \Upsilon' \rangle \lambda x.t'}
\end{align*}
\begin{align*}
\tag{Abs} \inference 
{}
{\langle \Phi \rangle \lambda x . t \Downarrow \langle \Phi \rangle \lambda x.t}
\end{align*}
\begin{align*}
\tag{App} \inference
{\langle \Phi \rangle t_l \Downarrow \langle \Psi \rangle \lambda 
x.t_n \\ \langle \Psi, x' \mapsto t_m \rangle [x'/x]t_n \Downarrow \langle
\Upsilon \rangle \lambda y.t'}
{\langle \Phi \rangle t_l \; t_m \Downarrow \langle \Upsilon \rangle \lambda y.t'}
\end{align*}
\caption{Ariola et. al's Operational Semantics (Fixed)}
\label{fig:cbn_fixed}
\end{figure}

\subsection{Related Work}

Chlipala implements a compiler from a STLC to a simple instruction machine in
\cite{chlipala2007certified}. In many ways it is more sophisticated than our 
work: it converts to CPS, performs closure conversion, and proves a similar
compiler correctness theorem to the one we've proved here. The primary
difference is that we've defined a call-by-need compiler, which forces us to
reason about updating thunks in the heap, a challenge not faced by
call-by-value implementations.

Breitner formalizes Launchbury's natural semantics and proves an optimization is
sound with respect to the semantics \cite{launchburynatural,breitnerthesis}. By
relating his formalization with ours, these projects could be combined to prove
a more sophisticated lazy compiler correct: one with non-trivial optimizations
applied.

CakeML \cite{cakeml14} is a verified compiler for a large subset of the Standard
ML language formalized in HOL4 \cite{slind2008brief}. Like Chlipala's work, this
is a call-by-value language, though they prove correctness down to an x86
machine model, and are working with a much larger real-world source language.
They also make divergence arguments along the lines of \cite{functionalbigstep},
strengthening their correctness theorem in the presence of nontermination. It's
also worth noting that like \cite{leroy2012compcert}, they are also formalizing
a front end to the compiler. 

As part of the DeepSpec project, Weirich et al. have been working on formalizing
Haskell's core semantics \cite{weirich2017spec, spector2018total}. There is
opportunity to use my work in combination with the DeepSpec project to implement
and verify a full-featured Haskell compiler.

