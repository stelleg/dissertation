\section{Conclusion} \label{sec:conc}

Lazy evaluation has long suffered from high overhead when delaying computations.
While strictness analysis helps to alleviate this issue, there are and there
always will be cases it cannot catch. Existing implementations choose to pay an
up-front cost of constructing a flat environment to ensure efficient variable
lookup if a delayed computation is used. When a delayed computation is never
used, this overhead is wasted. In this paper we have presented a novel approach
that minimizes this overhead. We have achieved this overhead minimization by
taking an old idea, shared environments, and using them in a novel way. By
leveraging the structure inherent in a shared environment to share results of
computation, we have avoided some of the overheads involved in delaying a
computation. 

We conclude by summarizing the key points of this paper. First, a shared
environment, explicitly represented as a cactus stack, is a natural way to share
the results of computation as required by lazy evaluation. Second, this approach
is in a sense \emph{lazier} about lazy evaluation than existing implementations
because it avoids some unnecessary packaging. Third, this approach can be
formalized in both big-step and small-step semantics. Lastly, the abstract
machine can be implemented as a compiler in a straightforward way, yielding
performance comparable to existing implementations. 
