\label{sec:introduction}
As discussed in Chapter~\ref{chap:intro}, lazy functional programming languages
like Haskell lend themselves particularly well to reasoning about correctness.
It is for this reason we are motivated to build a verified compiler for a
call-by-need semantics, the default semantics for lazy functional languages: we
want this reasoning to be preserved.  Unfortunately, one of the challenges for
formalization of non-strict compilers is that the semantics of call-by-need
abstract machines tend to be complex, incorporating complex optimizations into
the semantics, requiring preprocessing of terms, and closures of variable sizes
\cite{jonesstg, TIM}.  In this chapter, we use the \ce machine, taking advantage
of its simplicity to ease the formal reasoning burden that goes with building a
verified compiler.

Verified compilers provide powerful guarantees about the code they generate and
its relation to the corresponding source code \cite{chlipala2007certified,
leroy2012compcert, cakeml14}.  In particular, for higher order functional
languages, they ensure that the non-trivial task of compiling lambda
calculus and its extensions to machine code is implemented correctly,
preserving source semantics. The return on investment for verified compilers is
high: any reasoning about any program which is compiled with a verified compiler
is provably preserved. 

Existing verified compilers have focused on call-by-value semantics
\cite{chlipala2007certified, leroy2012compcert, cakeml14}. This semantics has
the property of being historically easier to implement than call-by-need, and
therefore likely easier to reason about formally. This chapter formalizes the
\ce machine described in Chapter 2. The Coq proof assistant~\cite{barras1997coq}
is used to implement and prove the correctness of our compiler. We start with a
source language of $\lambda$ calculus with de Bruijn indices: 
\begin{align*}
 t &::= t \; t \; | \; x \; | \;  \lambda \; t \\
 x &\in \mathbb{N}
\end{align*}
The source semantics is the big-step operational semantics of the \ce 
machine, which uses shared environments to share results between instances of a
bound variable. To strengthen the result, and relate it to a better-known
semantics, we also show that the call-by-name \ce machine implements Curien's
call-by-name calculus of closures. 

It may surprise the reader to see that we do not start with a better known
call-by-need semantics; this concern is addressed in
Section~\ref{sec:discussion}. The proof of compiler correctness, along with the
proof that our call-by-name semantics implements Curien's semantics, hopefully
convinces the reader that we have indeed implemented a call-by-need semantics,
despite not using a better known definition of call-by-need. 

For our target, we define a simple instruction machine, described in
Section~\ref{sec:im_semantics}. This simple target allows us to describe the
compiler and proofs concisely for the chapter, while still allowing
flexibility in eventually verifying a compiler down to machine code for some
set of real hardware, e.g., x86, ARM, or Power. 

Our main result is a proof that whenever the source semantics evaluates to a
value, the compiled code evaluates to the same value. While there are stronger
definitions of what qualifies as a verified compiler, We argue that this is
sufficient in Section~\ref{sec:discussion}. This main result, along with the
proof that the call-by-name version of our semantics implements Curien's
calculus of closures, are the primary contributions of this chapter. 

The chapter is structured as follows. In Section~\ref{sec:cem_big} we describe
the source syntax and semantics (the big-step \ce semantics) in detail.  We also
use this section to define a call-by-name version of the semantics, and show
that it implements Curien's calculus of closures \cite{curien1991abstract}.  In
Section~\ref{sec:cem_small} we describe the small-step \ce semantics and its
relation to the big-step semantics. In Section~\ref{sec:im_semantics} we describe
the instruction machine syntax and semantics. In Section~\ref{sec:compiler} we
describe the compilation from machine terms to assembly language. In
Section~\ref{sec:correctness} we describe how the evaluation of compiled programs
is related to the small-step \ce semantics. We compose this proof with the proof
that the small-step semantics implement the big-step semantics to show that the
instruction machine implements the big-step semantics. In
Section~\ref{sec:discussion} we discuss threats to validity, future work, and
related work. The Coq source code with all the definitions and proofs described
in this chapter is available at \url{https://github.com/stelleg/cem\_coq}. 
