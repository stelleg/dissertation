\subsection{Results} \label{sec:res}

Figure~\ref{fig:res} gives the benchmark results.  In general, the compiler is
outperformed by GHC, sometimes significantly, and outperforms UHC. I spend the
remainder of the section analyzing these performance differences.

\begin{figure*}
\centering
\begin{tabularx}{\textwidth}{l | X | X | X | X | X}
& $\mathcal{\mathcal{C} \mskip -4mu \mathcal{E}}$ & GHC -O0 & UHC -O0 & GHC -O3 & UHC -O3 \\
\hline
\texttt{exp3 8} & 1.530 & 1.176 & 3.318 & 1.038 & 2.286 \\
\texttt{tak 16 8 0} & 0.366 & 0.146 & 1.510 & 0.006 & 1.416 \\
\texttt{primes 1500} & 0.256 & 0.272 & 1.518 & 0.230 & 1.532 \\
\texttt{queens 9} & 0.206 & 0.050 & 0.600 & 0.012 & 0.598 \\
\texttt{fib 35} & 2.234 & 0.872 & 10.000 & 0.110 & 8.342 \\
\texttt{digits-of-e1 1000} & 3.576 & 1.274 & 21.938 & 0.118 & 22.010 \\
\texttt{digits-of-e2 1000} & 0.404 & 0.792 & 3.430 & 0.372 & 3.278 \\
\texttt{fannkuch 8} & 0.560 & 0.084 & 2.184 & 0.048 & 2.196 \\
\end{tabularx}
\caption{Machine Literals Benchmark Results. Measurement is wall clock time,
units are seconds. Times averaged over 5 runs.}
\label{fig:res}
\end{figure*}

There are many optimizations built into the abstract machine underlying GHC,
but profiling indicates that three in particular lead to much of the performance
disparity: 

\begin{itemize}
\item \textbf{Register allocation:} The $\mathcal{\mathcal{C} \mskip -4mu \mathcal{E}}$ machine has no register
allocator. In contrast, by passing arguments to functions in registers, GHC
avoids much heap thrashing.
\item \textbf{Unpacked literals:} This allows GHC to keep machine literals
without tags in registers for tight loops. In contrast, the $\mathcal{\mathcal{C} \mskip -4mu \mathcal{E}}$
machine operates entirely on the stack, and has a code pointer associated with
every machine literal. 
\item \textbf{Y combinator:} Because recursion in the $\mathcal{\mathcal{C} \mskip -4mu \mathcal{E}}$ machine is
implemented with a Y combinator, it performs poorly. This could be alleviated
with CFA-based techniques, similar to those used in ~\cite{rozas1992taming}. 
\end{itemize}

Lack of register allocation is the primary current limitation of the $\mathcal{\mathcal{C} \mskip -4mu \mathcal{E}}$
machine. The STG machine pulls the free variables into registers, allowing tight
loops with everything kept in registers. However, it is less clear how to
effectively allocate registers in a fully shared environment setting.
That said, I believe being lazier about register allocation, e.g., not loading
values into registers that may not be used, could have some performance benefits.

To isolate the effect of register allocation and unpacked machine
literals, I replace machine integers with Church numerals in a compatible
subset of the evaluation programs. Figure~\ref{fig:res-church}
shows the performance results with this modification, which are much improved,
with the $\mathcal{\mathcal{C} \mskip -4mu \mathcal{E}}$ machine occasionally
even outperforming optimized GHC.

\begin{figure*}
\centering
\begin{tabularx}{\textwidth}{l | X | X | X | X | X}
& $\mathcal{\mathcal{C} \mskip -4mu \mathcal{E}}$ & GHC -O0 & UHC -O0 & GHC -O3 & UHC -O3 \\
\hline
\texttt{tak 14 7 0} & 1.610 & 2.428 & 7.936 & 1.016 & 7.782 \\
\texttt{primes 32} & 0.846 & 1.494 & 4.778 & 0.666 & 5.290 \\
\texttt{queens 8} & 0.242 & 0.374 & 1.510 & 0.154 & 1.508 \\
\texttt{fib 23} & 0.626 & 0.940 & 5.026 & 0.468 & 5.336 \\
\texttt{digits-of-e2 6} & 0.138 & 1.478 & 5.056 & 0.670 & 5.534 \\
\texttt{fannkuch 7} & 0.142 & 0.124 & 0.796 & 0.040 & 0.808 \\
\end{tabularx}
\caption{Church Numeral Benchmark Results. Measurement is wall clock time, 
units are seconds. Times averaged over 5 runs.}
\label{fig:res-church}
\end{figure*}

Next, I consider the disparity due to the Y-combinator, by running a simple
exponentiation example with Church numerals, calculating $3^8 - 3^8 = 0$. In
this case, the $\mathcal{\mathcal{C} \mskip -4mu \mathcal{E}}$ machine
significantly outperforms both GHC and UHC, as seen in
Figure~\ref{fig:res-church-exp} .

\begin{figure*}
\begin{tabularx}{\textwidth}{l | X | X | X | X | X}
& $\mathcal{\mathcal{C} \mskip -4mu \mathcal{E}}$ & GHC -O0 & UHC -O0 & GHC -O3 & UHC -O3 \\
\hline
\texttt{pow 3 8} & 0.564 & 1.994 & 4.912 & 0.906 & 4.932 \\
\end{tabularx}
\caption{Church Numeral Exponentiation Benchmark Results. Measurement is wall clock time, 
units are seconds. Times averaged over 5 runs.}
\label{fig:res-church-exp}
\end{figure*}

These results give us confidence that by adding the optimizations mentioned
above, among others, the $\mathcal{\mathcal{C} \mskip -4mu \mathcal{E}}$ machine has the potential to be the
basis of a real-world compiler. I discuss how some of these optimizations can
be applied to the $\mathcal{\mathcal{C} \mskip -4mu \mathcal{E}}$ machine in Section~\ref{sec:disc}.

\subsection{The Cost of the Cactus}

Recall that variable lookup is linear in the index of the variable, following
pointers until the index is zero. As one might guess, the lookup cost is high.
For example, for the queens benchmark without any optimizations, variable lookup
took roughly $80-90\%$ of the $\mathcal{\mathcal{C} \mskip -4mu \mathcal{E}}$ machine runtime, as measured
by profiling. Much of that cost was for lookups of known combinators, however,
so for the benchmarks above I added the inlining mentioned in the previous
section. Still, even with this simple optimization, variable lookup takes
roughly $50\%$ of execution time. There is some variation across benchmarks, but
this is a rough approximation for the average cost. I discuss how this cost
could be addressed in future work in Section~\ref{sec:disc}.


