\section{Big-Step \ce} \label{sec:cem_big}

This section reviews our big-step source semantics. A big-step semantics
has the advantage of powerful, easy-to-use induction properties. This eases
reasoning about many program properties. We will also review the small-step
semantics and prove that it implements the big-step semantics, but by showing
that our implementation preserves the big-step semantics, we prove preservation
of any inductive reasoning on the structure of evaluation tree.  

As in Chapter~\ref{chap:ce}, our source syntax is the lambda calculus with de
Bruijn indices. De Bruijn indices count the number of intermediate lambdas
between the occurrence of the variable and its binding lambda.  
\begin{align*}
 t &::= t \; t \; | \; x \; | \;  \lambda \; t \\
 x &\in \mathbb{N}
\end{align*}
The essence of the \ce semantics (Figure~\ref{fig:bigstep}) is that it
implements a shared environment, and uses its structure to share results of
computations. This makes possible a simple abstract machine that operates on the
lambda calculus directly, which is uncommon among call-by-need abstract machines
\cite{jonesstg,launchburynatural,TIM,johnsson1984efficient}. This simplifies
formalization, as we do not need to prove that intermediate transformations,
e.g., lambda lifting, are semantics-preserving. Another advantage of the \ce
machine is that it has constant-sized closures, obviating the need to reason
about re-allocating the results of computation and adding indirections required
because of closure size changes from thunk to value \cite{jonesstg}. We operate
on closures, which combine terms with pointers into the shared environment,
which is implemented as a heap. Every heap location contains a cell, which
consists of a closure and a pointer to the next environment location, which I
will refer to as the environment continuation.  Variable dereferences index into
this shared environment structure, and if/when a dereferenced location evaluates
to a value, the original closure (potentially a thunk or closure not evaluated
to WHNF) will be replaced with that value. The binding of a new variable extends
the shared environment structure with a new
cell. This occurs during application, which evaluates the left hand side to an
abstraction, then extends the environment with the argument term closed under the
environment pointer of the application. The App rule ensures that two
variables bound to the same argument closure will point to the same location in
the shared environment. Because they point to the same location by construction
of the shared environment, we can update that location with the value computed
at the first variable dereference, and then each subsequent dereference will
point to this value. The variable rule applies the update by indexing into the
shared environment structure and replacing the closure at that location with the
resulting value. It is worth noting that while the closures in the heap cells
are mutable, the shared environment structure is never mutated. This property is
crucial when reasoning about variable dereferences. The $\mu\left(l, i\right)$
function looks up a variable index in the shared environment structure by
following environment continuation pointers, returning the location and cell
pointed to by the final step. See the Coq source
for a formal treatment. Note that we require that fresh heap locations are
greater than zero. This is required for reasoning about compilation to the
instruction machine, which we will return to in Section~\ref{sec:im_semantics}.
While here we constrain fresh heap locations to be fresh with respect to the
entire heap domain, for a real implementation, this is far too strong a
constraint, as it doesn't allow any sort of heap re-use. We return to this issue
in Section~\ref{sec:discussion}, and discuss how this could be relaxed to either
allow reasoning about garbage collection or direct heap reuse.

The fact that our natural semantics is defined on the lambda calculus with de Bruijn
indices differs from most existing definitions of call-by-need, such as
Ariola's call-by-need \cite{ariola1995call} or Launchbury's lazy semantics
\cite{launchburynatural}. These semantics are defined on the lambda calculus with named
variables. While it should be possible to relate our semantics to
these, the comparison is certainly made more difficult by this
disparity.\footnote{Both of these well known existing semantics have known
problems that arise during formalization, as discussed in
Section~\ref{sec:discussion}.} A more fruitful relation to semantics operating
on the lambda calculus with named variables would likely be relating Curien's
calculus of closures to call-by-name semantics implemented with substitution. We
return to this discussion in Section~\ref{sec:discussion}.

As mentioned in Section~\ref{chap:ce}, these big-step semantics do not
explicitly include a notion of nontermination. Instead, nontermination would be
implied by the negation of the existence of an evaluation relation. This
prevents reasoning directly about nontermination in an inductive way, but for
the purpose of our primary theorem this is acceptable. 

One interesting property of defining an inductive evaluation relation in a
language such as Coq is that computation can be done on the evaluation tree. In
other words, the evaluation relation given above defines a data type, one that
computation can be done on in standard ways. For example, we could potentially
compute properties such as size and depth, which would be related to operational
properties of compiled code. 

Finally, given a term $t$, the initial configuration is defined as
$\left(t\left[0\right], \epsilon\right)$. As discussed, the choice of the null
pointer for the environment pointer is not completely arbitrary, but chosen
across our semantics uniformly to represent failed environment lookup. 

\subsection{Call-By-Name}

This section reviews the call-by-name variant of our big-step semantics and
provides a proof that it is an implementation of Curien's call-by-name calculus
of closures. 

See Figure~\ref{fig:bigstepname} for the definition of our call-by-name
semantics. Note that the only change from our call-by-need semantics is that we
do not update the heap location with the result of the dereferenced computation.
This is the essence of the difference between call-by-name and call-by-need.

Recall Figure~\ref{fig:curien} for a formalization of Curien's call-by-name
semantics. We define a heterogeneous equivalence relation between our shared
environment and Curien's environment. Effectively, this relation is the
proposition that the shared environment structure is a linked list
implementation of the environment list in Curien's semantics. This is defined
inductively, and we require that every closure reachable in the environment is
also equivalent.  We say two closures are equivalent if their terms are
identical and their environments are equivalent. 

Given these definitions, we can prove that the call-by-name \ce semantics
implements Curien's call by name semantics: 

\begin{thm}
If a closure $c$ in Curien's call-by-name semantics is equivalent to a
configuration $c'$, and $c$ steps to $v$, then there exists a $v'$ that our
call-by-name semantics steps to from $c'$ that is equivalent to $v$.
\end{thm}
\begin{proofoutline}
The proof proceeds by induction on Curien's step relation. The abstraction rule
is a trivial base case. The variable lookup rule uses a helper lemma that proves
by induction on the variable that if the two environments are equivalent and the
variable indexes to a closure, then the $\mu$ function will look up an
equivalent closure. The application rule uses a helper lemma which proves that a
fresh allocation will keep any equivalent environments equivalent, and that the
new environment defined by the fresh allocation will be equivalent to the
extended environment of Curien's semantics.
\end{proofoutline}

By proving that Curien's semantics is implemented by the call-by-name variant of
our semantics, we provide further evidence that our call-by-need is a meaningful
semantics. 

One important note is that nowhere do we require that a term being evaluated is
closed under its environment. Indeed, it's possible that a term with free variables
can be evaluated by both semantics to a value as long as a free variable is
never dereferenced. This theme will recur through the rest of the chapter, so it
is worth keeping in mind.  
