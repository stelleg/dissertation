\section{Discussion and Related Work} \label{sec:disc}

In this section I compare our the compiler presented in this chapter with
earlier work. I also discuss areas for future work.

\subsection{Closure Representation}
Appel and Shao \cite{shao1994space} and Appel and Jim \cite{appel1988optimizing}
both cover the design space for closure representation, and develop an approach
called \emph{safely linked closures}. It uses flat closures when
there is no duplication, and links in a way that preserves liveness, to prevent
violation of the \emph{safe for space complexity} (SSC) rule
\cite{shao1994space}. While I do not address SSC or garbage collection in
general, understanding the relationship between SSC and shared environment
call-by-need is an interesting area for future work. In particular, hot
environments with no sharing could benefit greatly from replacing shared
structure with flat.

\subsection{Eval/Apply vs. Push/Enter}
Marlow and Peyton Jones describe two approaches to the implementation of
function application: eval/apply, where the function is evaluated and then
passed the necessary arguments, and push/enter, where the arguments are pushed
onto the stack and the function code is entered \cite{marlow2006making}. They
conclude that despite push/enter being a standard approach to lazy machines,
eval/apply performs better. While our current approach uses push/enter,
investigating whether eval/apply could be usefully implemented for a shared
environment machine like the $\mathcal{\mathcal{C} \mskip -4mu \mathcal{E}}$
machine is an interesting avenue for future work.

\subsection{Collapsed Markers}
Friedman et al.\ show how a machine can be designed to prevent multiple adjacent
update markers being pushed onto the stack \cite{lkm}.  This property is
desirable because multiple adjacent update markers are always updated with the
same value. They give examples showing that in some cases, these redundant
update markers can cause an otherwise constant-space stack to grow unbounded.
They implement an optimization that collapses update markers by adding a layer
of indirection between heap locations and closures. I propose a similar
approach, but without the performance hit caused by an extra layer of
indirection. Upon a variable dereference the $\mathcal{\mathcal{C} \mskip -4mu
\mathcal{E}}$ machine checks if the top of the stack is an update. If it is,
instead of pushing a redundant update marker onto the stack, it replaces the
closure in the heap at the desired location with an update marker.  Then, the
variable dereference rule checks for an update marker upon dereference, and will
update accordingly. I have begun to implement this optimization, but leave the
full implementation and description for future work.

\subsection{Register Allocation} \label{sec:alloc}
One advantage of flat environments is that register allocation is
straightforward \cite{appel1992compiling,jonesstg,terei2010llvm}. It is less
obvious how to do register allocation with the $\mathcal{\mathcal{C} \mskip
-4mu \mathcal{E}}$ machine. I speculate that it should be possible to do a
sufficient job, particularly in the cases that matter most, e.g. unboxed
machine literals, though certainly not easy. 

One possible approach that could work well with our shared environment approach
would be to only load strict free variables into registers. That is to say, some
environment variables may not be used, so only the ones that will definitely be
used should be loaded into registers, while the rest should be loaded on demand. 

