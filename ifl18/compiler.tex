\section{Compiler} \label{sec:compiler}

In this section we describe the compiler, which compiles lambda terms with
de Bruijn indices to programs. The compiler proceeds by recursion on lambda
terms, keeping a current index into the program to ensure correct linking
without a separate pass. For variables, when we get to zero we push the current
environment pointer and a null instruction pointer to denote the update marker
to the location of the closure being entered. Then we \textit{mov} the closure
at that location into \textit{r1} and \textit{ep}, and \textit{jump} to
\textit{r1}, recalling that the \textit{jump} sets the \textit{ip}. For 
nonzero variables, we replicate traversing the environment pointer \textit{i}
times before loading the closure.  For applications, we calculate the program
location of the argument basic block, and push that and the current environment
pointer onto the stack, effectively pushing an argument closure on top of the
stack. We then \textit{jump} to the left hand side of the application, as is
standard for push-enter evaluation. For abstractions, we use a conditional jump
depending on whether the top of the stack is a null pointer (and therefore an
update marker) or a valid instruction pointer (and therefore an argument). If it
is an update marker, we update the heap location defined by the update marker
with the current value instruction pointer and the current environment pointer.
We must point to the first of the three abstractions basic blocks, as this value
could later update another heap location as well. In the case that the top of
the stack was a valid instruction pointer, we allocate a new chunk of 3 word of
memory, and \textit{mov} the argument closure into it, with the current
environment pointer as the environment continuation. We then set our current
environment pointer to this fresh location. This is the process by which we
extend our shared environment structure in the instruction machine. Finally, we
perform an unconditional jump to the next basic block, which is the first basic
block of the compiled body of the lambda. As this is an unconditional jump to
the next basic block, for real machine code this jump can be omitted. 

\begin{figure}
\begin{align*}
var \; 0 :=  \;
  &push \; ep : \\
  &push \; 0 : \\
  &mov \; \left(ep\%0\right) \; r1 : \\
  &mov \; \left(ep\%1\right) \; ep : \\
  &jump \; r1 \\
var \; \left(i+1\right) := \;
  &mov \; \left(ep\%2\right) \; ep : \\
  &var \; i \\
compile \; i \; k := \; & \left[var \; i\right] \\
compile \; \left( m \; n \right) \; k := \;
  &let \; ms = compile \; m \; \left(k+1\right) \; in \\
  &let \; nk = 1+k+length \; ms \; in \\
  &push \; ep :  \\
  &push \; nk : \\
  &jump \; \left(k+1\right) :: \\
  &ms \concat compile \; n \; nk \\
compile \; \left(\lambda b\right) \; k := \;
  &pop \; r1 : \\
  &jump \; \left(r1, k+1 \right) \; \left(k+2\right) :: \\
  &pop \; r1 : \\
  &mov \; k \; r1\%0 : \\
  &mov \; ep \; r1\%1 : \\
  &jump \; k :: \\
  &new \; 3 \; r2 : \\
  &mov \; r1 \; \left(r2\%0\right) : \\
  &pop \; \left(r2\%1\right) : \\
  &mov \; ep \; \left(r2\%2\right) : \\
  &mov \; r2 \; ep : \\
  &jump \; \left(k+3\right) :: \\
  &compile \; b \; \left(k+3\right)
\end{align*}
\caption{Compiler Definition}
\label{fig:compiler}
\end{figure}

Being able to define the full compiler this simply is crucial to this
verification project. Other, more sophisticated implementations of call-by-need,
such as the STG machine, are much harder to implement and reason about. It is
worth noting that despite this simplicity, initial tests suggest that performance
is not as horrible as one might suspect, and is often competitive with state of
the art \cite{cem}.

As with the relation discussed in Section~\ref{sec:cem_big}, note that even when
compiling, we do not require that a term is closed to compile it. Indeed, we
will happily generate code that if entered, will attempt to dereference the null
pointer, leaving the machine stuck. Because we are only concerned with proving
that we implement the source semantics in the case that it evaluates to a value,
this is not a problem. If we wanted to strengthen our proof further, we would
try to show that if the source semantics gets stuck trying to dereference a free
variable, the implementation would get stuck in the same way, both failing to
dereference a null pointer.    
