
Proof irrelevance is an important idea in philosophy of logic. It is the notion
that we don't really need to care how the proof was built, only that it was
sound. In machine checked proofs this notion is particularly relevant: if the
machine checks the proofs of our lemmas, we can be sure that they are valid
proofs.

In contrast, we \emph{must} care about the contents of the definitions and
theorems. Without them, the reader can't be sure what's been proved is what is
being claimed to have been proved. Therefore, we use this section to define and
discuss select Coq definitions and theorem statements in the implementation and
proof of correctness of the verified compiler. We attempt to convey what worked
well, and what posed significant challenges in the hope of informing future work
in this area. The formalization is available at
\url{https://github.com/stelleg/cem\_coq} and as of this dissertation has been
successfully type-checked with Coq version 8.9.  The version used for this
dissertation is tagged with \texttt{dissertation}, which at the time of writing
enables it to be downloaded as a GitHub release tarball. Inevitably, we will use
definitions from both the Coq standard libraries, or generic definitions
implemented throughout the project. In these cases, if the reader has questions,
I encourage them to refer to the source code. In general, this section attempts
to cover the most important definitions and theorems for the 

It's also worth noting that the reader will find many definitions, lemmas,
algorithms, and proofs in the repository that are unused in the final proof.
This is a byproduct of the process of proving. Often, when one starts a proof it
seems like a certain property or lemma will be required, so one proves that
lemma, only to discover later that it was not necessary. Such helper lemmas
sometimes come in useful later, during a different proof. It is for this reason
there are so much unused Coq code in the repository. Hopefully, if the project
is continued, some of them will prove useful.

Another point worth noting is that the Coq code makes heavy use of Unicode
characters and notations. In hindsight, the heavy use of fancy notations was
almost always a mistake, and a refactoring to remove them would likely improve
readability. The one forgivable exception is the occasional infix operator, for
which I make no apologies.

\section{Big-Step \ce Semantics}
Here we define the big step syntax and semantics, which we use as our input
language semantics. This is a formalization of the Figure~\ref{fig:bigstep}.

We start with the lambda calculus with deBruijn indices.

\begin{verbatim}
Inductive tm : Type := 
  | var : nat → tm 
  | lam : tm → tm
  | app : tm → tm → tm.
\end{verbatim}

Next, we define other helper definitions, including closures, environments
(\texttt{env}), cells, heaps, configurations, what doing a lookup into the
shared environment means (\texttt{clu}), and what replacing a closure at a
location in a heap means (\texttt{update}).

\begin{verbatim}
Definition env := nat.

Record closure : Type := close {
  cl_tm : tm;
  cl_en : env
}.

Record cell : Type := cl {
  cell_cl : closure;
  cell_env : env
}.

Definition heap := Map nat cell.

Record configuration : Type := conf {
  conf_h : heap;
  conf_c : closure
}.

Fixpoint clu (v env:nat) (h:heap) : option (nat * cell) := 
  match lookup env h with
  | None => None
  | Some (cl c a) => match v with
    | S n => clu n a h
    | 0 => Some (env, cl c a)
    end
  end.

Fixpoint update (h : heap) (l : env) (v : closure) : heap := match h with 
  | [] => []
  | (u, cl c e)::h => if beq_nat l u 
    then (u, cl v e) :: update h l v 
    else (u, cl c e) :: update h l v
  end.
\end{verbatim}
Finally, we define the actual machine big-step semantics.
\begin{verbatim}
Reserved Notation " c1 '⇓' c2 " (at level 50).
Inductive step : configuration → configuration → Type :=
  | Id : ∀ M x y z Φ Ψ v e, clu y e Φ = Some (x, {M, z}) → 
                    ⟨Φ⟩M ⇓ ⟨Ψ⟩v →
    ⟨Φ⟩close (var y) e ⇓ ⟨update Ψ x v⟩v
  | Abs : ∀ N Φ e, ⟨Φ⟩close (:λN) e ⇓ ⟨Φ⟩close (:λN) e
  | App : ∀ N M B B' Φ Ψ Υ f e ne ae, isfresh (domain Ψ) f → f > 0 →
          ⟨Φ⟩close M e ⇓ ⟨Ψ⟩close (:λB) ne → 
      ⟨Ψ, f ↦ {close N e, ne}⟩close B f ⇓ ⟨Υ⟩close (:λB') ae   →
              ⟨Φ⟩close (M@N) e ⇓ ⟨Υ⟩close (:λB') ae
where " c1 '⇓' c2 " := (step c1 c2).
\end{verbatim}

\section{Small-Step \ce Semantics}

Our small step semantics are a straightforward implementation of the big step
semantics. The language is the same lambda calculus with deBruijn indices, while
we introduce a stack to match marker updates and argument bindings to variables.
This is a formalization of Figure~\ref{fig:cesm}. 

Most of the syntax is shared with the big-step semantics, and is imported
directly from the big-step module. We share the definitions for stacks and
states, followed directly by the small-step semantics.
\begin{verbatim}
Definition stack := list (closure + nat).

Inductive state : Type := st {
  st_hp : heap; 
  st_st : stack;
  st_cl : closure
}.

Reserved Notation " c1 '→_s' c2 " (at level 50).
Inductive step : transition state :=
  | Upd : ∀ Φ b e l s, 
  st Φ (inr l::s) (close (lam b) e) →_s 
  st (update Φ l (close (lam b) e)) s (close (lam b) e)
  | Var : ∀ Φ s v l c e e', clu v e Φ = Some (l,cl c e') → 
  st Φ s (close (var v) e) →_s st Φ (inr l::s) c
  | Abs : ∀ Φ b e f c s, isfresh (domain Φ) f → f > 0 → 
  st Φ (inl c::s) (close (lam b) e) →_s st ((f, cl c e):: Φ) s (close b f)
  | App : ∀ Φ e s n m, 
  st Φ s (close (app m n) e) →_s st Φ (inl (close n e)::s) (close m e)
where " c1 '→_s' c2 " := (step c1 c2).
\end{verbatim}

\subsection{Relation to Big-Step}

We prove that the small step semantics implement the big step semantics with the
following lemma. The description of the proof can be found in
Section~\ref{sec:cem_cesm}. 

\begin{verbatim}
Notation " c1 '→_s*' c2 " := 
  (refl_trans_clos cesm.step c1 c2) (at level 30). 

Lemma cem_cesm : ∀ Φ Ψ c v, 
  conf Φ c ⇓ conf Ψ v → ∀ s, 
  st Φ s c →_s* st Ψ s v. 
\end{verbatim}

\section{Instruction Machine} 

Finally, we change representations and describe the assembly language of the
abstract instruction machine; the target of the verified compiler. This is the
formalization of language defined in Figure~\ref{fig:im_syntax}. Note the use of
coercions to help ease the use of type-safe read and write operands, and an
infix operator to duplicate common syntax for real world assembly languages when
indexing an offset. Note that for our purposes, we only ever need a constant
offset, but one could easily extend the language to allow for offsets to be
defined by a read operand.  

\begin{verbatim}
Definition Word := nat.
Definition Ptr := nat.

Inductive Reg := 
  | IP
  | EP
  | R1
  | R2.

Inductive WO := 
  | WR : Reg → WO
  | WM : Reg → nat → WO.

Coercion WR : Reg >-> WO.
Infix "%" := WM (at level 30).

Inductive RO := 
  | RW : WO → RO
  | RC : nat → RO.

Coercion RW : WO >-> RO.
Coercion RC : nat >-> RO.

Inductive Instr : Type :=
  | push : RO → Instr
  | pop : WO → Instr
  | new : nat → WO → Instr 
  | mov : RO → WO → Instr.

Inductive BasicBlock : Type :=
  | instr : Instr → BasicBlock → BasicBlock
  | jump : option (RO*Ptr) → RO → BasicBlock.

Definition Program := list BasicBlock.
\end{verbatim}

Finally, note that we use a list of basic blocks as our program. This list is
indexed by integers in the machine semantics, so translating this to a Harvard 
architecture machine semantics should be relatively straightforward. 

Now that we have the language for our abstract instruction machine target, we
can look at the formal semantics for it. Without loss of generality, we choose a
step relation on instructions, wrapped by a step instruction on full basic
blocks. This eases reasoning about the relation to the small step \ce
relation. This is a formalization of the semantics described in
Figure~\ref{fig:im_semantics}. 

We start with a our register file and machine state, along with a
straightforward semantics for reading and writing from operands given a 
machine state. We omit some uninteresting helper functions. 

\begin{verbatim}
Inductive RegisterFile := mkrf {
  ip : Ptr;
  ep : Ptr;
  r1 : Ptr; 
  r2 : Ptr
}. 

Inductive State := st {
  st_rf : RegisterFile;
  st_p : Program;
  st_s : Stack;
  st_h  : Heap
}.

Open Scope nat_scope. 
Inductive read : RO → State → Ptr → Type :=
  | read_reg : ∀ r s, read (RW (WR r)) s (rff (st_rf s) r)
  | read_mem : ∀ r o rf p s h v, 
    lookup (o+rff rf r) h = Some v →
    read (RW (WM r o)) (st rf p s h) v
  | read_const : ∀ c s, read (RC c) s c.

Inductive write : WO → Word → State → State → Type :=
  | write_reg : ∀ r rf p s h w, 
    write (WR r) w (st rf p s h) (st (upd r w rf) p s h)
  | write_mem : ∀ r o rf p s h w, 
    write (WM r o) w (st rf p s h) 
                     (st rf p s (replace beq_nat (o+rff rf r) w h)).
\end{verbatim}

Given our machine definitions and read and write semantics, we can move directly
to defining the step relation for instructions. The \texttt{step\_bb} relation
defines the execution of a full basic block by induction on the instructions
contained in that basic blocks. The \texttt{step} relation then wraps that
relation and defines how the state changes given execution of the basic block
pointed to by the instruction pointer. Note that we don't define an explicit
halting relation. Instead, the machine will be stuck when the current code to be
executed is a value and the stack is empty, which is where the small step
\ce semantics also get stuck. In contrast, the native code implementation has a
check to ensure the stack is non-empty. A full compiler with a sufficiently
sophisticated type system could likely do away with this check, only terminating
with the \texttt{main} function returning a value of type \texttt{World}. 

\begin{verbatim}
Inductive step_bb : BasicBlock → State → State → Type := 
  | step_push : ∀ rf p s h ro is v sn,
  read ro (st rf p s h) v → 
  step_bb is (st rf p (v::s) h) sn → 
  step_bb (instr (push ro) is) (st rf p s h) sn
  | step_pop : ∀ rf p s h wo is w s' sn,
  write wo w (st rf p s h) s' → 
  step_bb is s' sn →
  step_bb (instr (pop wo) is) (st rf p (w::s) h) sn
  | step_new : ∀ rf p s h wo is w s' n sn,
  (∀ i, i < n → not (In (i+w) (domain h))) →
  w > 0 →
  write wo w (st rf p s (zeroes n w ++ h)) s' →
  step_bb is s' sn → 
  step_bb (instr (new n wo) is) (st rf p s h) sn 
  | step_mov : ∀ s is ro wo s' v sn, 
  read ro s v → write wo v s s' → 
  step_bb is s' sn → 
  step_bb (instr (mov ro wo) is) s sn
  | step_jump0 : ∀ ro k j s s', 
  read ro s 0 →
  write (WR IP) k s s' → 
  step_bb (jump (Some (ro, k)) j) s s'
  | step_jumpS : ∀ ro k j s s' l k', 
  l > 0 →
  read ro s l →
  read j s k → 
  write (WR IP) k s s' → 
  step_bb (jump (Some (ro, k')) j) s s'
  | step_jump : ∀ ro s s' l, 
  read ro s l →
  write (WR IP) l s s' → 
  step_bb (jump None ro) s s'
.

Inductive step : transition State :=
  | enter : ∀ rf p s h k bb sn,
    read IP (st rf p s h) k → 
    nth_error p k = Some bb → 
    step_bb bb (st rf p s h) sn →
    step (st rf p s h) sn.
\end{verbatim}

With our step function defined, we can start thinking about how we want to
implement our compiler, and how to relate the instruction machine to the small
step semantics of our input language.  

\subsection{Relation to Small-Step Semantics}

We now have everything we need, except for the compiler definition, which we
will address in the next section. We start by assuming a \texttt{new} function,
which is effectively a \texttt{malloc} that given a heap, returns a non-null
free contiguous region of memory. Of course, in reality \texttt{malloc} is a
partial function, but we aren't modelling running out of memory in this work. 

\begin{verbatim}
Variable new : ∀ (n:nat) (h : im.Heap), sigT (λ w:nat, 
  prod (∀ i, lt i n → (i+w) ∉ domain h)
       (w > 0)).

Definition prog_eq (p : Ptr) (pr : Program) (t : tm) := 
  let subpr := assemble t p in subpr = firstn (length subpr) (skipn p pr).
\end{verbatim}

Next, we discuss heap relations between the instruction machine and the \ce \\
machine semantics. Unfortunately, this gets incredibly involved. Despite much
effort, I was unable to find a more elegant relation between the two.

\begin{verbatim}
Inductive heap_rel : cesm.heap → im.Heap → Type := 
  | heap_nil : heap_rel [] [] 
  | heap_cons : ∀ l l' ne ch ih ip ep ine e t, 
    l ∉ domain ch → l' ∉ domain ih → 
    S l' ∉ domain ih → S (S l') ∉ domain ih →
    l > 0 → l' > 0 → 
    heap_rel ch ih → 
    heap_rel 
      ((l, cl (close t e) ne)::ch) 
      ((l', ip)::(S l', ep)::(S (S l'), ine)::ih).

Fixpoint in_heap_rel (ch : cesm.heap) (ih : im.Heap)
                     (r : heap_rel ch ih)
                     (l e ne : nat) (t : db.tm) 
                     (il ip ep nep : Ptr) : Type := match r with
  | heap_nil => False
  | heap_cons l' il' ne' cht iht ip' ep' nep' e' t' _ _ _ _ _ _ rt => 
    if andb (beq_nat l l') (beq_nat il il') then
    (ne' = ne) *  (ip' = ip) * (ep' = ep) * 
    (nep' = nep) * (e' = e) * (t' = t)
    else 
      if andb (negb (beq_nat l l')) (negb (beq_nat il il'))
        then in_heap_rel cht iht rt l e ne t il ip ep nep
        else False
  end.

Inductive env_eq (ch : cesm.heap) (ih : im.Heap) (r : heap_rel ch ih)
  : nat → Ptr → Type :=
  | e0 : env_eq ch ih r 0 0 
  | eS : ∀ l e ne t il ip ep nep, 
    in_heap_rel ch ih r l e ne t il ip ep nep →
    env_eq ch ih r l il.

Inductive heap_eq (ch : cesm.heap) (ih : im.Heap) 
                  (r : heap_rel ch ih) (p : Program) : Type := 
  | mkheap_eq : 
    (∀ l e ne t il ip ep nep, 
      in_heap_rel ch ih r l e ne t il ip ep nep →
      (prog_eq ip p t) * 
      (env_eq ch ih r e ep) * 
      (env_eq ch ih r ne nep)) → 
    heap_eq ch ih r p.

\end{verbatim}

In words, the \texttt{heap\_rel} defines a heap relation that simply relates an
entry in the \ce heap to the three machine words in the instruction machine
heap. \texttt{in\_heap\_rel} is a decidable relation that determines whether or
not a given heap location and cell at that location and their corresponding
machine analogues are related by a heap relation object. We say environments are
equal if they are either both the empty environment or the first element of the
environments are in the heap relation, and the tails are equal. Two heaps are
equivalent if for every cell, the term portions are equivalent and the
environment and environment continuations are equivalent. The inductive relation
on the environments is crucial for inductive reasoning on deBruijn indices.

Given our heap and environment relations, we can move to notions of closure,
stack, and complete state equivalences.

\begin{verbatim}
Inductive clos_eq (ch : cem.heap) (ih : im.Heap) 
                  (r : heap_rel ch ih) (p : Program):
                  closure → Ptr → Ptr → Type :=
  | c_eq : ∀ t e ip ep, 
           prog_eq ip p t → 
           env_eq ch ih r e ep →
           clos_eq ch ih r p (cem.close t e) ip ep. 

Inductive stack_eq (ch : cem.heap) (ih : im.Heap) 
                   (r : heap_rel ch ih) (p : Program) : 
  cesm.stack → im.Stack → Type := 
  | stack_nil : stack_eq ch ih r p nil nil
  | stack_upd : ∀ l e ne t il ip ie ine cs is, 
                in_heap_rel ch ih r l e ne t il ip ie ine →
                stack_eq ch ih r p cs is →
                stack_eq ch ih r p (inr l::cs) (0::il::is)
  | stack_arg : ∀ ip ep cs is c, 
                 ip > 0 →
                 clos_eq ch ih r p c ip ep →
                 stack_eq ch ih r p cs is → 
                 stack_eq ch ih r p (inl c::cs) (ip::ep::is).

Inductive state_rel (cs : cesm.state) (is : im.State) : Type := 
  | str : ∀ r, 
  heap_eq (st_hp cs) (st_h is) r (st_p is) →
  clos_eq (st_hp cs) (st_h is) r (st_p is) (st_cl cs) 
          (rff (st_rf is) IP) (rff (st_rf is) EP) →
  stack_eq (st_hp cs) (st_h is) r (st_p is) (st_st cs) (st_s is) →
  state_rel cs is.
\end{verbatim}
For the stack relation, we have either the empty stacks or two stacks with
either both update markers or both argument closures on top, and equivalent
stacks below that. For update markers, we have that the two marker locations
exist in the heap relation. For argument closures, we have that the closures are
equivalent. 

The closure and state relations are straightforward, the closure relation
requires that the term and subprogram are equal, and that the environments are
equivalent. The state relation requires equivalent closure and registerfile
entries (IP and EP), equivalent heaps, and equivalent stacks.

Finally, we can pose our lemma that the instruction machine implements the
small-step semantics.

\begin{verbatim}
Lemma cesm_im : ∀ v s s', state_rel s s' → 
  cesm.step s v → 
  sigT (λ v', prod (refl_trans_clos im.step s' v') (state_rel v v')).
\end{verbatim}
This states that if we have related small step and instruction machine states,
and we take a small step, then the instruction machine will take a step to a
state that is related to the new state of the small-step machine.  While we
don't discuss the proof in detail, it's worth mentioning that even with many
supporting definitions and lemmas, it took on the order of 2000 Ltac statements
to prove. Note that is a single step, but the reflexive transitive closure
version follows trivially. 

\section{Compiler and Correctness}

We've seen the \texttt{assemble} function referred to relate the lambda terms to
the machine instructions in the previous section. This sections defines that
function, which is the entire compiler implementation.

\begin{verbatim}
Infix ";" := instr (at level 30, right associativity).

Fixpoint var_inst (i : nat) : BasicBlock := match i with
  | 0 => push EP ;
         push (RC 0) ;
         mov (EP%0) R1 ;
         mov (EP%1) EP ;
         jump None R1
  | S i => mov (EP%2) EP ; 
           var_inst i
  end.


Fixpoint assemble (t : tm) (k : nat) : Program := match t with  
  | var v => [var_inst v]
  | app m n => let ms := assemble m (1+k) in
               let nk := 1+k+length ms in
                push EP ;
                push (RC nk) ;
                jump None (RC (1+k)) :: 
                ms ++ 
                assemble n nk
  | lam b => pop R1 ;
             jump (Some (RW (WR R1), (1+k))) (RC (2+k)) ::
             (*Update*)
             pop R1 ;  
             mov (RC k) (R1%0) ;
             mov EP (R1%1) ;
             jump None (RC k) ::
             (*Take*)
             new 3 R2 ;
             mov R1 (R2%0);
             pop (R2%1) ;
             mov EP (R2%2) ;
             mov R2 EP ;
             jump None (3+k) :: 
             assemble b (3+k)
  end. 
\end{verbatim}
We can see that the compiler is very simple, only requiring ~35 lines of code.
It is this simplicity that enables the formal reasoning achieved in the previous
section. 

Finally, we can define our top level correctness theorem.

\begin{verbatim}
Definition compile t := assemble t 0.

Theorem compile_correct (t : db.tm) v : cem.step (cem.I t) v → 
  sigT (λ v', refl_trans_clos im.step (im.I (compile t)) v' *
              state_rel (cesm.st (cem.conf_h v) nil (cem.conf_c v)) v').
\end{verbatim}
This theorem states that if a term steps in the big step semantics to a value,
then the instruction machine will step in zero or more steps to a related state.
Knowledge of the \texttt{state\_rel} relation is crucial here: it would be
trivial to define a meaningless relation, e.g. the relation defined by
\texttt{λ c i, True}, and prove the relation trivially. Because we know that
the relation requires equivalence of term and subprogram pointed to by IP, we
know that the theorem is what we want.

\section{Curien}

To relate our implementation to a known semantics, we choose Curien's calculus
of closures. We show that the call-by-name \ce semantics implement Curien's
calculus of closures. We start by defining Curien's calculus of closures. 

\begin{verbatim}
Inductive closure := | close : tm → list closure → closure. 
Definition env := list closure.

Inductive step : closure → closure → Type := 
  | Abs : ∀ b e, step (close (lam b) e) (close (lam b) e)
  | Var : ∀ x e v c, nth_error e x = Some c → step c v → step (close (var x) e) v
  | App : ∀ m n b e v mve, 
      step (close m e) (close (lam b) mve) → 
      step (close b (close n e::mve)) v → 
      step (close (app m n) e) v.
\end{verbatim}

This is a formalization of the semantics in Figure~\ref{fig:curien}. We relate
this to the call-by-name variant of the \ce semantics. Defined in
Figure~\ref{fig:bigstepname}, we formalize this semantics in Coq as follows.

\begin{verbatim}
Reserved Notation " c1 '⇓n' c2 " (at level 50).
Inductive step : configuration → configuration → Type :=
  | Id : ∀ M x z Φ Ψ y v e, clu y z Φ = Some (x, {M, e}) → 
            ⟨Φ⟩M ⇓n ⟨Ψ⟩v →
    ⟨Φ⟩close (var y) z ⇓n ⟨Ψ⟩v
  | Abs : ∀ N Φ e, ⟨Φ⟩close (:λN) e ⇓n ⟨Φ⟩close (:λN) e
  | App : ∀ N M B B' Φ Ψ Υ f e ne ae, isfresh (domain Ψ) f → 
          ⟨Φ⟩close M e ⇓ ⟨Ψ⟩close (:λB) ne → 
      ⟨Ψ, f ↦ {close N e, ne}⟩close B f ⇓ ⟨Υ⟩close (:λB') ae   →
              ⟨Φ⟩close (M@N) e ⇓ ⟨Υ⟩close (:λB') ae
where " c1 '⇓n' c2 " := (step c1 c2).
\end{verbatim}

Note the only difference between this semantics and the call-by-need is the lack
of an updated heap in the \texttt{Id} rule (and we don't require nonzero heap
locations, but this is unimportant). With these two semantics defined, we can
relate them by first relating environments and closures. We start by defining an
inductive relation on what it means for environments to be equivalent.
Effectively, a cactus environment is equivalent a Curien environment if
following the following the environment pointers results in equivalent closures
at each location. 

\begin{verbatim}
Inductive env_eq (h : heap) : curien.env → cem.env → Type := 
  | env_eq_nil : ∀ l, env_eq h nil l
  | env_eq_cons : ∀ t e e' l l' l'',
      lookup l h = Some (cl (cem.close t l') l'') →
      env_eq h e l' →
      env_eq h e' l'' →
      env_eq h (curien.close t e :: e') l
  .
Inductive close_eq (h : heap) : curien.closure → cem.closure → Type := 
  | close_equiv : ∀ t e l, 
      env_eq h e l → 
      close_eq h (curien.close t e) (cem.close t l).
\end{verbatim}

With this relation, we can define and prove a helper lemma that takes advantage
of the monotonicity of the cactus environment to implement a cactus environment
to prove that if an environment is equivalent to a Curien environment, and the
call-by-name \ce semantics take a step, it will stay equivalent. This is
noteworthy due to being one of the few places in the proof structures that would
need to be changed to incorporate a notion of heap reuse/garbage collection.

\begin{verbatim}
Lemma env_eq_step : ∀ h h' c v e l, 
  cem_name.step (conf h c) (conf h' v) →
  env_eq h e l → 
  env_eq h' e l.
\end{verbatim}

Finally, we are ready to define our correctness lemma.

\begin{verbatim}
Theorem step_eq : ∀ h c c' v, close_eq h c c' → curien.step c v →
  sigT (λ co, match co with conf h' v' => prod 
    (cem_name.step (conf h c') co)
    (close_eq h' v v') 
  end). 
\end{verbatim}

Note the use of \texttt{sigT} instead of standard \texttt{∃} syntax. This is due
to our use of the \texttt{Type} universe for all of our judgements, instead of
\texttt{Prop}. I do this to enable the eventual possibility of doing
computation on the judgements, to allow reasoning about time and space
requirements of the semantics. The impredicative nature of \texttt{Prop} was
never required in the proofs of this dissertation, so the use of \texttt{Type}
was sufficient. 
