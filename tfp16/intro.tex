Existing implementations of call-by-need take care in \emph{packaging} a delayed
computation, or \emph{thunk}, by building a closure with an array that contains
the bindings of all free variables \cite{jonesstg,boquist1997grin}. The overhead
induced by this operation is well known, and is one reason existing
implementations avoid thunks wherever possible \cite{johnsson1984efficient}. The
key insight of our Cactus Environment (\ce) Machine is that this overhead can be
minimized by only recording a location in a shared environment.

As an example, consider the application $f \; e$. In existing call-by-need
implementations, e.g., the STG machine\cite{jonesstg}, a closure with a flat
environment will be constructed for $e$.  Doing so incurs a time and memory cost
proportional to the number of free variables of $e$. \footnote{In some
implementations, these are lambda-lifted to be formal parameters, but the
principle is the same.} I minimize this packaging cost by recording a
location in a shared environment, which requires only two
machine words (and two instructions) for the thunk: one for the code pointer,
and one for the environment pointer. One way to think about the approach is that
it is \emph{lazier} about lazy evaluation: in the case that $e$ is unneeded, the
work to package it in a thunk is entirely wasted. In the spirit of lazy
evaluation, I attempt to minimize this potentially unnecessary work.  

In this chapter I present a simple implementation of the \ce machine that
compiles a simple lazy functional language to x86\_64 assembly. In addition, it
presents a preliminary evaluation that shows performance comparable to existing
implementations (Sections~\ref{sec:impl} and~\ref{sec:eval}). 

I describe a straightforward implementation of \ce in Section~\ref{sec:impl},
extended with machine literals and primitive operations, and compiling directly
to native code. I evaluate the implementation in Section~\ref{sec:eval},
showing that it is capable of performing comparably to existing implementations
despite lacking several common optimizations, and I discuss the results. I
discuss related work, the limitations of our approach, and some ideas for future
work in Section~\ref{sec:disc}.
