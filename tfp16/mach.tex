\section{Small-Step \ce} \label{sec:mach}
Using the calculus of cactus environments defined in the previous section, we derive an abstract machine: the \ce machine. The syntax and semantics are
defined in Figure~\ref{fig:cesm}. 

\begin{figure}
\textbf{Syntax}
\begin{align*}
\tag{State} s &::= \langle c, \sigma, \mu \rangle \\
\tag{Term} t &::= i \; | \; \lambda t \; | \; t \; t  \\
\tag{Variable} i &\in \mathbb{N}  \\
\tag{Closure} c &::= t \left[l\right] \\
\tag{Value} v &::= \lambda t\left[l\right] \\
\tag{Heap} \mu &::= \epsilon \; | \; \mu \left[ l \mapsto \rho \right] \\
\tag{Environment} \rho &::= \bullet \; | \; c \cdot l \\
\tag{Stack} \sigma &::= \square \; | \; \sigma \; c \;  | \; \sigma \; u \\
\tag{Location} l,u,f &\in \mathbb{N}
\end{align*}
\textbf{Semantics}
\begin{align*}
\tag{Upd}
\langle v,  \sigma \; u , \mu \rangle 
  &\rightarrow
\langle v, \sigma, \mu\left(u \mapsto v \cdot l\right) \rangle  
\; \textnormal{where} \; c \cdot l = \mu\left(u\right) \\
\tag{Lam}
\langle \lambda t\left[l\right], \sigma \; c, \mu \rangle 
  &\rightarrow
\langle t\left[f\right], \sigma, \mu\left[f \mapsto c \cdot l\right]\rangle f
\not \in \textnormal{dom}\left(\mu\right)  \\
\tag{App}
\langle t \; t'\left[l\right], \sigma, \mu \rangle
  &\rightarrow
\langle t\left[l\right], \sigma \; t'\left[l\right], \mu \rangle \\
\tag{Var}
\langle i\left[l\right], \sigma, \mu \rangle
  &\rightarrow
\langle c, \sigma \; l'', \mu \rangle
\; \textnormal{where} \; l'' \mapsto c \cdot l' = \mu\left(l, i\right)
\end{align*}
\caption{Small-step \ce syntax and semantics}
\label{fig:cesm}
\end{figure}

The small-step semantics operate identically to the big-step, extended only with
a context to implement the updates from the Id subderivation ($\sigma \; u$) and
the operands from the App subderivation ($\sigma \; c$).  Much like the
calculus, a term $t$ is inserted into an initial state $\langle t[0], \sigma,
\epsilon[0\mapsto\bullet]\rangle$ . On the update rule, the current closure is a
value, and there is an update marker as the outermost context.  This implies
that a variable was entered and that the current closure represents the
corresponding value for that variable. Thus, we update the location $u$ that the
variable entered, replacing whatever term was entered with the current closure.
The Lam rule takes an argument off the context and binds it to a variable,
allocating a fresh heap location for the bound variable. This ensures that every
instance of the variable will point to this location, and thus the bound term
will be evaluated at most once. The App rule simply pushes an argument term in
the current environment. The Var rule enters the closure pointed to by the
\textit{i}'th environment location.  

To get some intuition for the $\mathcal{\mathcal{C} \mskip -4mu \mathcal{E}}$
machine and how it works, consider Figure~\ref{fig:state}, evaluation of the
term $(\lambda a.(\lambda b.b \; a) \lambda c.c \; a) \; ((\lambda i.i)
\lambda j.j)$, or $(\lambda(\lambda0\;1)\;\lambda0\;1)\;((\lambda0)\;
\lambda0)$ with deBruijn indices.

\begin{sidewaysfigure}
\begin{align*}
&\langle (\lambda(\lambda0\;1)\;\lambda0\;1)\;((\lambda0)\;\lambda0)[0],\square,\epsilon[0\mapsto\bullet]\rangle\\
&\rightarrow_{\mathcal{\mathcal{C} \mskip -4mu \mathcal{E}}}\langle \lambda(\lambda0\;1)\;\lambda0\;1[0],\square (\lambda0)\;\lambda0[0],\epsilon[0\mapsto\bullet]\rangle\\ 
&\rightarrow_{\mathcal{\mathcal{C} \mskip -4mu \mathcal{E}}}\langle (\lambda0\;1)\;\lambda0\;1[1],\square,\epsilon[0\mapsto\bullet][1\mapsto(\lambda0)\;\lambda0[0]\cdot0]\rangle\\ 
&\rightarrow_{\mathcal{\mathcal{C} \mskip -4mu \mathcal{E}}}\langle \lambda0\;1[1],\square \lambda0\;1[1],\epsilon[0\mapsto\bullet][1\mapsto(\lambda0)\;\lambda0[0]\cdot0]\rangle\\ 
&\rightarrow_{\mathcal{\mathcal{C} \mskip -4mu \mathcal{E}}}\langle 0\;1[2],\square,\epsilon[0\mapsto\bullet][1\mapsto(\lambda0)\;\lambda0[0]\cdot0][2\mapsto\lambda0\;1[1]\cdot1]\rangle\\ 
&\rightarrow_{\mathcal{\mathcal{C} \mskip -4mu \mathcal{E}}}\langle 0[2],\square 1[2],\epsilon[0\mapsto\bullet][1\mapsto(\lambda0)\;\lambda0[0]\cdot0][2\mapsto\lambda0\;1[1]\cdot1]\rangle\\ 
&\rightarrow_{\mathcal{\mathcal{C} \mskip -4mu \mathcal{E}}}\langle \lambda0\;1[1],\square 1[2] 2,\epsilon[0\mapsto\bullet][1\mapsto(\lambda0)\;\lambda0[0]\cdot0][2\mapsto\lambda0\;1[1]\cdot1]\rangle\\ 
&\rightarrow_{\mathcal{\mathcal{C} \mskip -4mu \mathcal{E}}}\langle \lambda0\;1[1],\square 1[2],\epsilon[0\mapsto\bullet][1\mapsto(\lambda0)\;\lambda0[0]\cdot0][2\mapsto\lambda0\;1[1]\cdot1]\rangle\\ 
&\rightarrow_{\mathcal{\mathcal{C} \mskip -4mu \mathcal{E}}}\langle 0\;1[3],\square,\epsilon[0\mapsto\bullet][1\mapsto(\lambda0)\;\lambda0[0]\cdot0][2\mapsto\lambda0\;1[1]\cdot1][3\mapsto1[2]\cdot1]\rangle\\ 
&\rightarrow_{\mathcal{\mathcal{C} \mskip -4mu \mathcal{E}}}\langle 0[3],\square 1[3],\epsilon[0\mapsto\bullet][1\mapsto(\lambda0)\;\lambda0[0]\cdot0][2\mapsto\lambda0\;1[1]\cdot1][3\mapsto1[2]\cdot1]\rangle\\ 
&\rightarrow_{\mathcal{\mathcal{C} \mskip -4mu \mathcal{E}}}\langle 1[2],\square 1[3] 3,\epsilon[0\mapsto\bullet][1\mapsto(\lambda0)\;\lambda0[0]\cdot0][2\mapsto\lambda0\;1[1]\cdot1][3\mapsto1[2]\cdot1]\rangle\\ 
&\rightarrow_{\mathcal{\mathcal{C} \mskip -4mu \mathcal{E}}}\langle 0[1],\square 1[3] 3,\epsilon[0\mapsto\bullet][1\mapsto(\lambda0)\;\lambda0[0]\cdot0][2\mapsto\lambda0\;1[1]\cdot1][3\mapsto1[2]\cdot1]\rangle\\ 
\end{align*}
\caption{$\mathcal{\mathcal{C} \mskip -4mu \mathcal{E}}$ machine example.
Evaluation of $(\lambda(\lambda0\;1)\;\lambda0\;1)\;((\lambda0)\;\lambda0)$}
\label{fig:state}
\end{sidewaysfigure}
\begin{sidewaysfigure}
\begin{align*}
&\rightarrow_{\mathcal{\mathcal{C} \mskip -4mu \mathcal{E}}}\langle (\lambda0)\;\lambda0[0],\square 1[3] 3 1,\epsilon[0\mapsto\bullet][1\mapsto(\lambda0)\;\lambda0[0]\cdot0][2\mapsto\lambda0\;1[1]\cdot1][3\mapsto1[2]\cdot1]\rangle\\ 
&\rightarrow_{\mathcal{\mathcal{C} \mskip -4mu \mathcal{E}}}\langle \lambda0[0],\square 1[3] 3 1 \lambda0[0],\epsilon[0\mapsto\bullet][1\mapsto(\lambda0)\;\lambda0[0]\cdot0][2\mapsto\lambda0\;1[1]\cdot1][3\mapsto1[2]\cdot1]\rangle\\ 
&\rightarrow_{\mathcal{\mathcal{C} \mskip -4mu \mathcal{E}}}\langle 0[4],\square 1[3] 3 1,\epsilon[0\mapsto\bullet][1\mapsto(\lambda0)\;\lambda0[0]\cdot0][2\mapsto\lambda0\;1[1]\cdot1][3\mapsto1[2]\cdot1][4\mapsto\lambda0[0]\cdot0]\rangle\\ 
&\rightarrow_{\mathcal{\mathcal{C} \mskip -4mu \mathcal{E}}}\langle \lambda0[0],\square 1[3] 3 1 4,\epsilon[0\mapsto\bullet][1\mapsto(\lambda0)\;\lambda0[0]\cdot0][2\mapsto\lambda0\;1[1]\cdot1][3\mapsto1[2]\cdot1][4\mapsto\lambda0[0]\cdot0]\rangle\\ 
&\rightarrow_{\mathcal{\mathcal{C} \mskip -4mu \mathcal{E}}}\langle \lambda0[0],\square 1[3] 3 1,\epsilon[0\mapsto\bullet][1\mapsto(\lambda0)\;\lambda0[0]\cdot0][2\mapsto\lambda0\;1[1]\cdot1][3\mapsto1[2]\cdot1][4\mapsto\lambda0[0]\cdot0]\rangle\\ 
&\rightarrow_{\mathcal{\mathcal{C} \mskip -4mu \mathcal{E}}}\langle \lambda0[0],\square 1[3] 3,\epsilon[0\mapsto\bullet][1\mapsto\lambda0[0]\cdot0][2\mapsto\lambda0\;1[1]\cdot1][3\mapsto1[2]\cdot1][4\mapsto\lambda0[0]\cdot0]\rangle\\ 
&\rightarrow_{\mathcal{\mathcal{C} \mskip -4mu \mathcal{E}}}\langle \lambda0[0],\square 1[3],\epsilon[0\mapsto\bullet][1\mapsto\lambda0[0]\cdot0][2\mapsto\lambda0\;1[1]\cdot1][3\mapsto\lambda0[0]\cdot1][4\mapsto\lambda0[0]\cdot0]\rangle\\ 
&\rightarrow_{\mathcal{\mathcal{C} \mskip -4mu \mathcal{E}}}\langle 0[5],\square,\epsilon[0\mapsto\bullet][1\mapsto\lambda0[0]\cdot0][2\mapsto\lambda0\;1[1]\cdot1][3\mapsto\lambda0[0]\cdot1][4\mapsto\lambda0[0]\cdot0][5\mapsto1[3]\cdot0]\rangle\\ 
&\rightarrow_{\mathcal{\mathcal{C} \mskip -4mu \mathcal{E}}}\langle 1[3],\square 5,\epsilon[0\mapsto\bullet][1\mapsto\lambda0[0]\cdot0][2\mapsto\lambda0\;1[1]\cdot1][3\mapsto\lambda0[0]\cdot1][4\mapsto\lambda0[0]\cdot0][5\mapsto1[3]\cdot0]\rangle\\ 
&\rightarrow_{\mathcal{\mathcal{C} \mskip -4mu \mathcal{E}}}\langle 0[1],\square 5,\epsilon[0\mapsto\bullet][1\mapsto\lambda0[0]\cdot0][2\mapsto\lambda0\;1[1]\cdot1][3\mapsto\lambda0[0]\cdot1][4\mapsto\lambda0[0]\cdot0][5\mapsto1[3]\cdot0]\rangle\\ 
&\rightarrow_{\mathcal{\mathcal{C} \mskip -4mu \mathcal{E}}}\langle \lambda0[0],\square 5 1,\epsilon[0\mapsto\bullet][1\mapsto\lambda0[0]\cdot0][2\mapsto\lambda0\;1[1]\cdot1][3\mapsto\lambda0[0]\cdot1][4\mapsto\lambda0[0]\cdot0][5\mapsto1[3]\cdot0]\rangle\\ 
&\rightarrow_{\mathcal{\mathcal{C} \mskip -4mu \mathcal{E}}}\langle \lambda0[0],\square 5,\epsilon[0\mapsto\bullet][1\mapsto\lambda0[0]\cdot0][2\mapsto\lambda0\;1[1]\cdot1][3\mapsto\lambda0[0]\cdot1][4\mapsto\lambda0[0]\cdot0][5\mapsto1[3]\cdot0]\rangle\\ 
&\rightarrow_{\mathcal{\mathcal{C} \mskip -4mu \mathcal{E}}}\langle \lambda0[0],\square,\epsilon[0\mapsto\bullet][1\mapsto\lambda0[0]\cdot0][2\mapsto\lambda0\;1[1]\cdot1][3\mapsto\lambda0[0]\cdot1][4\mapsto\lambda0[0]\cdot0][5\mapsto\lambda0[0]\cdot0]\rangle
\end{align*}
\caption{$\mathcal{\mathcal{C} \mskip -4mu \mathcal{E}}$ machine example.
Evaluation of $(\lambda(\lambda0\;1)\;\lambda0\;1)\;((\lambda0)\;\lambda0)$
(continued)}
\label{fig:state}
\end{sidewaysfigure}

