\section{Small-Step \ce} \label{sec:cem_small}

In this section I review the small-step semantics of the \ce 
machine, and show that it implements the big-step semantics of
Section~\ref{sec:cem_big}. This is a fairly straightforward transformation implemented
by adding a stack. The source language is the same, and I simply add a stack to
our configuration (and call it a state). The stack elements are either argument
closures or update markers. Update markers are pushed onto the stack when a
variable dereferences that location in the heap. When they are popped by an
abstraction, the closure at that location is replaced by said abstraction, so
that later dereferences by the same variable in the same scope dereference the
value, and do not repeat the computation.  Argument closures are pushed onto the
stack by applications, with the same environment pointer duplicated in the
current closure and the argument closure.  Argument closures are popped off the
stack by abstractions, which allocate a fresh memory location, write the
argument closure to it, write the environment continuation as the current
environment pointer, then enter the body of the abstraction with the fresh
environment pointer. This is the mechanism used for extending the shared
environment structure. The semantics is defined formally in
Figure~\ref{fig:cesm}.  

\subsection{Relation to Big Step} \label{sec:cem_cesm}
Here I prove that the small-step semantics implements the big-step semantics of
Section~\ref{sec:cem_big}. This requires first a notion of reflexive transitive closure,
which I define in the standard way. We also make use of the fact that the
reflexive transitive closure can be defined equivalently to extend from the left
or right. 

\begin{lemma}
If the big-step semantics evaluates from one configuration to another, then the
reflexive transitive closure of the small-step semantics evaluates from the same
starting configuration with any stack to the same value configuration with that
same stack.
\end{lemma}
\begin{proofoutline}
The proof proceeds by induction on the big-step relation. We define our
induction hypothesis so that it holds for all stacks, which gives us the
desired case of the empty stack as a simple specialization. The rule for
abstractions is the trivial base case. Var rule applies as the first step, and
the induction hypothesis applies to the stack with the update marker on it. To
ensure that the Upd rule applies I use the fact that the big-step semantics
only evaluates to abstraction configurations, and the fact that the reflexive
transitive closure can be rewritten with steps on the right. For the Application
rule, I take advantage of the fact that one can append two evaluations together,
as well as extend a reflexive transitive closure from the left or the right. As
with the Var rule I use the fact that the induction rule is defined for all
stacks to ensure evaluation of the left hand side to a value with the argument
on the top of the stack. Finally, I extend the environment with the argument
closure, and evaluate the result to a value by the second induction hypothesis.
\end{proofoutline}

Adding a stack in this fashion is a standard approach to converting between big
step and small-step semantics, and applies here in a straightforward way. 
